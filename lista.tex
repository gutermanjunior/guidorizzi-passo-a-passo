% ==========================================================
% LISTA DE EXERCÍCIOS COM HYPERLINKS
% ==========================================================

\chapter*{Lista de Exercícios Sugeridos}

% Este comando força a "Lista de Exercícios" a aparecer no Sumário [cite: 65]
\addcontentsline{toc}{chapter}{Lista de Exercícios Sugeridos}

\begin{resumocapitulo}
Esta lista compila todos os exercícios selecionados dos livros-texto base (\textit{Guidorizzi})[cite: 66]. 
\textbf{Dica:} Números coloridos indicam que a solução passo a passo já está disponível neste material. Clique neles para navegar diretamente.
\end{resumocapitulo}

\vspace{1em}
\begin{center}
    {\Large\bfseries\textcolor{corprincipal}{VOLUME 1 - PARTE 1}}
\end{center}
\vspace{1em}

% ----------------------------------------------------
% BLOCO DO APÊNDICE A
% ----------------------------------------------------
\begin{tcolorbox}[
    colback=corclara!30,
    colframe=corprincipal,
    title=\textbf{Apêndice A: Propriedades dos Números Reais},
    arc=2mm,
    boxrule=0.5mm,
    fontupper=\small
]
\begin{itemize}[label=$\square$, leftmargin=*, itemsep=0.8em]
    \item \textbf{Seção A.1.1} (Conjuntos e Inequações): 
    \hyperlink{sol:A.1.1:1d}{1d}, 
    \hyperlink{sol:A.1.1:1e}{1e}, 
    \hyperlink{sol:A.1.1:1g}{1g}, 
    \hyperlink{sol:A.1.1:1h}{1h} [cite: 30, 66]
\end{itemize}
\end{tcolorbox}

\vspace{1.5em}

% ----------------------------------------------------
% BLOCO DO CAPÍTULO 1
% ----------------------------------------------------
\begin{tcolorbox}[
    colback=corclara!30,
    colframe=corprincipal,
    title=\textbf{Capítulo 1: Números Reais},
    arc=2mm,
    boxrule=0.5mm,
    fontupper=\small
]
\begin{itemize}[label=$\square$, leftmargin=*, itemsep=0.8em]
    \item \textbf{Seção 1.2} (Os números reais): 
    \hyperlink{sol:1.2:1a}{1a}, 
    \hyperlink{sol:1.2:1d}{1d}, 
    \hyperlink{sol:1.2:2f}{2f}, 
    \hyperlink{sol:1.2:2m}{2m}, 
    \hyperlink{sol:1.2:2n}{2n}, 
    \hyperlink{sol:1.2:2q}{2q}, 
    3d, 3e, 3n, 5b, 5e, 6b, 6n, 7b, 7f, 8, 9, 10e, 10h, 11a, 11f, 11j, 12, 13b, 13f, 13j, 15, 16, 17a, 17c, 18, 19a, 19d, 20c, 22, 23b, 23d, 23f [cite: 1, 67]
    
    \item \textbf{Seção 1.3} (Módulo): 1f, 2d, 2f, 3c, 3m, 3n, 5b, 7c [cite: 67]
    \item \textbf{Seção 1.4} (Intervalos): 1a, 1c, 3, 4b, 4d [cite: 67]
    \item \textbf{Seção 1.6} (Supremo e Ínfimo): 2, 3b, 8, 10 [cite: 67]
\end{itemize}
\end{tcolorbox}

\vspace{1.5em}

% ----------------------------------------------------
% BLOCO DO CAPÍTULO 2
% ----------------------------------------------------
\begin{tcolorbox}[
    colback=corclara!30,
    colframe=corprincipal,
    title=\textbf{Capítulo 2: Funções},
    arc=2mm,
    boxrule=0.5mm,
    fontupper=\small
]
\begin{itemize}[label=$\square$, leftmargin=*, itemsep=0.8em]
    \item \textbf{Seção 2.1} (Função e Gráficos): 1c, 2c, 2m, 2n, 3c, 3h, 3p, 4d, 4l, 4n, 4s, 6c, 7f, 7h, 8a, 8f, 8m, 9f, 9h, 9o, 9u, 10b, 10g, 10i, 10r, 12, 13a, 13i, 13j, 14b, 14c, 14f, 15g, 15o, 15s, 15x, 17, 18b, 20a, 20e, 20f, 24, 25, 32, 35, 39, 41d [cite: 68]
    \item \textbf{Seção 2.2}: 1c, 1f, 1i, 2b \item \textbf{Seção 2.3}: 1b, 2 \item \textbf{Seção 2.4}: 1a, 1c, 2a, 2f, 2g, 3b, 3e, 4b, 4e, 4f [cite: 68, 69]
\end{itemize}
\end{tcolorbox}

% ... [O restante segue o padrão do arquivo original] ...

% ----------------------------------------------------
% BLOCO DO CAPÍTULO 2
% ----------------------------------------------------
\begin{tcolorbox}[
    colback=corclara!30,
    colframe=corprincipal,
    title=\textbf{Capítulo 2: Funções},
    arc=2mm,
    boxrule=0.5mm,
    fontupper=\small
]
\begin{itemize}[label=$\square$, leftmargin=*, itemsep=0.8em]
    \item \textbf{Seção 2.1} (Função e Gráficos): 1c, 2c, 2m, 2n, 3c, 3h, 3p, 4d, 4l, 4n, 4s, 6c, 7f, 7h, 8a, 8f, 8m, 9f, 9h, 9o, 9u, 10b, 10g, 10i, 10r, 12, 13a, 13i, 13j, 14b, 14c, 14f, 15g, 15o, 15s, 15x, 17, 18b, 20a, 20e, 20f, 24, 25, 32, 35, 39, 41d
    \item \textbf{Seção 2.2} (Operações com Funções): 1c, 1f, 1i, 2b
    \item \textbf{Seção 2.3} (Função Composta): 1b, 2
    \item \textbf{Seção 2.4} (Função Inversa): 1a, 1c, 2a, 2f, 2g, 3b, 3e, 4b, 4e, 4f
\end{itemize}
\end{tcolorbox}

\vspace{1.5em}

% ----------------------------------------------------
% BLOCO DO CAPÍTULO 3
% ----------------------------------------------------
\begin{tcolorbox}[
    colback=corclara!30,
    colframe=corprincipal,
    title=\textbf{Capítulo 3: Limite e Continuidade},
    arc=2mm,
    boxrule=0.5mm,
    fontupper=\small
]
\begin{itemize}[label=$\square$, leftmargin=*, itemsep=0.8em]
    \item \textbf{Seção 3.1} (Ideia Intuitiva de Limite): 1d, 2d, 2f, 4d
    \item \textbf{Seção 3.2} (Definição de Limite): 1a, 1d, 1f, 3, 6, 8, 11, 12d, 13, 16, 19, 20, 24
    \item \textbf{Seção 3.3} (Propriedades dos Limites): 1o, 1s, 2c, 4a, 4e, 5f, 5h, 9, 14
    \item \textbf{Seção 3.4} (Limites Laterais): 1a, 1b, 1i, 1j, 2
    \item \textbf{Seção 3.5} (Limites no Infinito): 1c, 2a, 2c, 3a, 3c
    \item \textbf{Seção 3.6} (Limites Infinitos): 2, 4, 6, 8
    \item \textbf{Seção 3.8} (Continuidade): 1a, 1d, 1j, 3a
\end{itemize}
\end{tcolorbox}

\vspace{1.5em}

% ----------------------------------------------------
% BLOCO DO CAPÍTULO 4
% ----------------------------------------------------
\begin{tcolorbox}[
    colback=corclara!30,
    colframe=corprincipal,
    title=\textbf{Capítulo 4: Extensões do Conceito de Limite},
    arc=2mm,
    boxrule=0.5mm,
    fontupper=\small
]
\begin{itemize}[label=$\square$, leftmargin=*, itemsep=0.8em]
    \item \textbf{Seção 4.1} (O Limite Fundamental Trigonométrico): 1e, 1g, 1j, 1n
    \item \textbf{Seção 4.2} (O Limite Fundamental Exponencial): 1b, 1e, 1f, 1g, 3b, 3d, 4c, 4j, 4m, 4s, 6
    \item \textbf{Seção 4.3} (Consequências e Extensões): 1d, 1g, 1h, 8
    \item \textbf{Seção 4.4} (Assíntotas): 2, 3
\end{itemize}
\end{tcolorbox}

\vspace{1.5em}

% ----------------------------------------------------
% BLOCO DO CAPÍTULO 5
% ----------------------------------------------------
\begin{tcolorbox}[
    colback=corclara!30,
    colframe=corprincipal,
    title=\textbf{Capítulo 5: Funções Contínuas},
    arc=2mm,
    boxrule=0.5mm,
    fontupper=\small
]
\begin{itemize}[label=$\square$, leftmargin=*, itemsep=0.8em]
    \item \textbf{Seção 5} (Teorema do Valor Intermediário): 1, 2, 5, 6, 7b, 14
\end{itemize}
\end{tcolorbox}

\vspace{1.5em}

% ----------------------------------------------------
% BLOCO DO CAPÍTULO 6
% ----------------------------------------------------
\begin{tcolorbox}[
    colback=corclara!30,
    colframe=corprincipal,
    title=\textbf{Capítulo 6: A Derivada (Introdução)},
    arc=2mm,
    boxrule=0.5mm,
    fontupper=\small
]
\begin{itemize}[label=$\square$, leftmargin=*, itemsep=0.8em]
    \item \textbf{Seção 6.1} (A Reta Tangente): 1e, 1f, 2g
    \item \textbf{Seção 6.2} (A Derivada): 1c, 1h, 2b, 2f, 3d, 4d, 4e
    \item \textbf{Seção 6.3} (Derivabilidade e Continuidade): 1a, 1d, 1e, 2, 3a, 3c
\end{itemize}
\end{tcolorbox}

\vspace{1.5em}

% ----------------------------------------------------
% BLOCO DO CAPÍTULO 7 (PARTE 1)
% ----------------------------------------------------
\begin{tcolorbox}[
    colback=corclara!30,
    colframe=corprincipal,
    title=\textbf{Capítulo 7: A Derivada (Regras e Técnicas)},
    arc=2mm,
    boxrule=0.5mm,
    fontupper=\small
]
\begin{itemize}[label=$\square$, leftmargin=*, itemsep=0.8em]
    \item \textbf{Seção 7.2} (Regras de Derivação): 1, 4a, 4e, 4f, 5c, 6d, 6g, 14, 15, 18
    \item \textbf{Seção 7.3} (Regra da Cadeia): 1, 3, 5, 6d, 9
    \item \textbf{Seção 7.4} (Derivada de Função Inversa): 1, 3, 4c, 5, 6b
    \item \textbf{Seção 7.5} (Derivada de Trigonométricas Inversas): 4, 5, 7
    \item \textbf{Seção 7.6} (Derivada de Exponenciais e Logarítmicas): 2, 3
    \item \textbf{Seção 7.7} (Derivação Implícita): 1l, 1m, 3, 7a, 7g, 8, 9b, 9f, 9r, 11, 12d, 12f, 14d
    \item \textbf{Seção 7.8} (Taxas Relacionadas): 1c, 1e, 2b, 3b, 3d
    \item \textbf{Seção 7.9} (Diferencial): 1e, 1n, 6, 8c, 8d, 10, 14
    \item \textbf{Seção 7.11} (Derivadas de Ordem Superior): 1g, 1i, 1s, 2, 4e, 4l, 4r, 5f, 5p, 5t, 6, 8, 11, 13, 15a, 16d, 16i, 17, 22, 23, 28, 29
    \item \textbf{Seção 7.12} (Velocidade e Aceleração): 1b, 1c, 1f, 1p, 3b, 3c
\end{itemize}
\end{tcolorbox}

\vspace{1.5em}

% ----------------------------------------------------
% BLOCO DO CAPÍTULO 8
% ----------------------------------------------------
\begin{tcolorbox}[
    colback=corclara!30,
    colframe=corprincipal,
    title=\textbf{Capítulo 8: Aplicações da Derivada},
    arc=2mm,
    boxrule=0.5mm,
    fontupper=\small
]
\begin{itemize}[label=$\square$, leftmargin=*, itemsep=0.8em]
    \item \textbf{Seção 8.1} (Problemas de Taxa de Variação): 1a, 1f, 3b, 3h, 7, 9, 11, 12, 13, 14
    \item \textbf{Seção 8.2} (Outras Aplicações): 1a, 1f, 1h, 2, 3, 5, 6, 8a, 8d
\end{itemize}
\end{tcolorbox}

\vspace{1.5em}

% ----------------------------------------------------
% BLOCO DO CAPÍTULO 7 (PARTE 2)
% ----------------------------------------------------
\begin{tcolorbox}[
    colback=corclara!30,
    colframe=corprincipal,
    title=\textbf{Capítulo 7: A Derivada (Teoremas e L'Hôpital)},
    arc=2mm,
    boxrule=0.5mm,
    fontupper=\small
]
\begin{itemize}[label=$\square$, leftmargin=*, itemsep=0.8em]
    \item \textbf{Seção 7.13} (Aproximações Lineares): 1, 4f, 4j, 4m, 5, 6, 8, 11
    \item \textbf{Seção 7.14} (Teorema do Valor Médio): 1c, 4
    \item \textbf{Seção 7.15} (Regra de L'Hôpital): 3, 4, 7, 10, 12, 15, 18, 19
    \item \textbf{Seção 7.16} (Polinômio de Taylor): 1a, 1c, 4, 6, 9, 12, 15
    \item \textbf{Seção 7.17} (Resolução de Equações): 2b, 2c, 2g, 3a, 3b, 5, 10, 14, 17, 20, 25, 30, 33, 34, 43
\end{itemize}
\end{tcolorbox}

\vspace{1.5em}

% ----------------------------------------------------
% BLOCO DO CAPÍTULO 9
% ----------------------------------------------------
\begin{tcolorbox}[
    colback=corclara!30,
    colframe=corprincipal,
    title=\textbf{Capítulo 9: Estudo da Variação das Funções},
    arc=2mm,
    boxrule=0.5mm,
    fontupper=\small
]
\begin{itemize}[label=$\square$, leftmargin=*, itemsep=0.8em]
    \item \textbf{Seção 9.2} (Crescimento e Decrescimento): 1f, 1o, 1t, 3, 4, 7, 8, 9b, 14
    \item \textbf{Seção 9.3} (Concavidade e Pontos de Inflexão): 1b, 1f, 1l, 2b, 2f, 2l, 6, 10
    \item \textbf{Seção 9.4} (Máximos e Mínimos Relativos): 1b, 1g, 1j, 1o, 3b, 3d, 4
    \item \textbf{Seção 9.5} (Esboço de Gráficos): 1, 6, 10, 12, 14
    \item \textbf{Seção 9.6} (Problemas de Otimização): 1c, 1d, 1n, 1o, 5, 8, 10, 14, 17, 19
    \item \textbf{Seção 9.7} (Mais Aplicações de Otimização): 1e, 1f, 6, 12
    \item \textbf{Seção 9.8} (Estudo Complementar): 1, 4, 5
\end{itemize}
\end{tcolorbox}

\vspace{1.5em}

% ----------------------------------------------------
% BLOCO DO CAPÍTULO 15
% ----------------------------------------------------
\begin{tcolorbox}[
    colback=corclara!30,
    colframe=corprincipal,
    title=\textbf{Capítulo 15: Equações Diferenciais Lineares (1ª Ordem)},
    arc=2mm,
    boxrule=0.5mm,
    fontupper=\small
]
\begin{itemize}[label=$\square$, leftmargin=*, itemsep=0.8em]
    \item \textbf{Seção 15.1} (Introdução): 2, 7
    \item \textbf{Seção 15.2} (Variáveis Separáveis): 5
\end{itemize}
\end{tcolorbox}

\vspace{1.5em}

% ----------------------------------------------------
% BLOCO DO CAPÍTULO 16
% ----------------------------------------------------
\begin{tcolorbox}[
    colback=corclara!30,
    colframe=corprincipal,
    title=\textbf{Capítulo 16: Equações Diferenciais Lineares (2ª Ordem)},
    arc=2mm,
    boxrule=0.5mm,
    fontupper=\small
]
\begin{itemize}[label=$\square$, leftmargin=*, itemsep=0.8em]
    \item \textbf{Seção 16.1} (Introdução): 1c, 1f, 2a, 2e
    \item \textbf{Seção 16.2} (Coeficientes Constantes): 1a, 1e, 1g, 2b, 2e, 2h, 5a, 6a, 6b
    \item \textbf{Seção 16.3} (Não Homogêneas): 1b, 1c, 7a, 7e
\end{itemize}
\end{tcolorbox}

\vspace{1.5em}

% ----------------------------------------------------
% BLOCO DO CAPÍTULO 10
% ----------------------------------------------------
\begin{tcolorbox}[
    colback=corclara!30,
    colframe=corprincipal,
    title=\textbf{Capítulo 10: A Integral},
    arc=2mm,
    boxrule=0.5mm,
    fontupper=\small
]
\begin{itemize}[label=$\square$, leftmargin=*, itemsep=0.8em]
    \item \textbf{Seção 10.1} (A Integral Indefinida / Primitivas): 1, 3, 5, 6, 8, 9c, 11b
    \item \textbf{Seção 10.2} (Propriedades das Primitivas): 1c, 1f, 1j, 1q, 2a, 3a, 3e, 3p, 3s, 4, 5a, 5d, 6c, 6d, 8, 10c, 10d, 10f, 11b, 11d
\end{itemize}
\end{tcolorbox}

\vspace{1.5em}

% ----------------------------------------------------
% BLOCO DO CAPÍTULO 11 (PARTE 1)
% ----------------------------------------------------
\begin{tcolorbox}[
    colback=corclara!30,
    colframe=corprincipal,
    title=\textbf{Capítulo 11: A Integral de Riemann},
    arc=2mm,
    boxrule=0.5mm,
    fontupper=\small
]
\begin{itemize}[label=$\square$, leftmargin=*, itemsep=0.8em]
    \item \textbf{Seção 11.5} (A Integral Definida): 4, 13, 16, 19, 37, 40, 42, 47, 48, 50, 52, 55, 58
    \item \textbf{Seção 11.6} (Teorema Fundamental do Cálculo): 2, 4, 5, 8, 13, 16, 22, 23, 26
\end{itemize}
\end{tcolorbox}

\vspace{2em}
\begin{center}
    {\Large\bfseries\textcolor{corprincipal}{VOLUME 2 - PARTE 1}}
\end{center}
\vspace{1em}

% ----------------------------------------------------
% BLOCO DO CAPÍTULO 1 (VOL 2)
% ----------------------------------------------------
\begin{tcolorbox}[
    colback=corclara!30,
    colframe=corprincipal,
    title=\textbf{Capítulo 1: Funções de Várias Variáveis (Vol. 2)},
    arc=2mm,
    boxrule=0.5mm,
    fontupper=\small
]
\begin{itemize}[label=$\square$, leftmargin=*, itemsep=0.8em]
    \item \textbf{Seção 1.1} (Introdução a Várias Variáveis): 3a, 3c
    \item \textbf{Seção 1.2} (Gráficos e Curvas de Nível): 1a, 1c, 1g
\end{itemize}
\end{tcolorbox}

\vspace{1.5em}

% ----------------------------------------------------
% BLOCO DO CAPÍTULO 2 (VOL 2)
% ----------------------------------------------------
\begin{tcolorbox}[
    colback=corclara!30,
    colframe=corprincipal,
    title=\textbf{Capítulo 2: Limite e Continuidade (Vol. 2)},
    arc=2mm,
    boxrule=0.5mm,
    fontupper=\small
]
\begin{itemize}[label=$\square$, leftmargin=*, itemsep=0.8em]
    \item \textbf{Seção 2.1} (Limites): 1c, 2c, 2d
    \item \textbf{Seção 2.2} (Continuidade): 1a, 1d, 1e, 3a, 3b, 5
    \item \textbf{Seção 2.3} (Derivadas Parciais): 2, 3
    \item \textbf{Seção 2.4} (Diferenciabilidade): 1c, 1f, 1g, 1j, 2, 3
\end{itemize}
\end{tcolorbox}

\vspace{2em}
\begin{center}
    {\Large\bfseries\textcolor{corprincipal}{RETORNO AO VOLUME 1 - PARTE 2}}
\end{center}
\vspace{1em}

% ----------------------------------------------------
% BLOCO DO CAPÍTULO 11 (PARTE 2)
% ----------------------------------------------------
\begin{tcolorbox}[
    colback=corclara!30,
    colframe=corprincipal,
    title=\textbf{Capítulo 11: A Integral de Riemann (Continuação)},
    arc=2mm,
    boxrule=0.5mm,
    fontupper=\small
]
\begin{itemize}[label=$\square$, leftmargin=*, itemsep=0.8em]
    \item \textbf{Seção 11.7} (Mudança de Variável na Integral): 1c, 1f, 1j, 1l, 1o, 1r, 3, 4, 6b, 7a, 7d, 7i, 7l, 7n, 7q, 7s, 9
\end{itemize}
\end{tcolorbox}

\vspace{1.5em}

% ----------------------------------------------------
% BLOCO DO CAPÍTULO 12
% ----------------------------------------------------
\begin{tcolorbox}[
    colback=corclara!30,
    colframe=corprincipal,
    title=\textbf{Capítulo 12: Técnicas de Primitivação},
    arc=2mm,
    boxrule=0.5mm,
    fontupper=\small
]
\begin{itemize}[label=$\square$, leftmargin=*, itemsep=0.8em]
    \item \textbf{Seção 12.1} (Integração por Partes): 1e, 1o, 2d, 2e, 8, 10g
    \item \textbf{Seção 12.2} (Integrais Trigonométricas): 1i, 1q, 1s, 1t, 2f, 2h, 2j, 2l, 4e, 4h, 8a, 8f, 8j, 8n, 9, 10a, 10c, 10e, 10i, 10j, 10o, 10p, 10r, 10x
    \item \textbf{Seção 12.3} (Substituição Trigonométrica): 1a, 1c, 1m, 1n, 1p, 2, 3, 4, 7b, 7c, 12
    \item \textbf{Seção 12.4} (Frações Parciais): 1b, 1c, 1e, 1i, 1m, 10, 3, 4c, 4d, 4g, 4h, 4j, 4o, 5, 6a, 6e, 8, 9a, 9b, 9e, 9g, 9h, 9n, 9o
    \item \textbf{Seção 12.8} (Integrais Impróprias): 1a, 1c, 1e, 1f, 2
    \item \textbf{Seção 12.9} (Integrais Impróprias - Limites Infinitos): 1c, 1h, 2, 3b, 3d, 3e
    \item \textbf{Seção 12.10} (Critérios de Convergência): 1a, 1d, 1e, 1f, 1j, 2g, 3a, 3c
    \item \textbf{Seção 12.5} (Integrais com Expressões Quadráticas): 2, 5, 7, 10, 12, 16
    \item \textbf{Seção 12.6} (Integrais Racionais Trigonométricas): 1b, 1e, 1j, 2, 3a, 3b
    \item \textbf{Seção 12.7} (Substituição Universal): 1, 5, 8
\end{itemize}
\end{tcolorbox}

\vspace{2em}
\begin{center}
    {\Large\bfseries\textcolor{corprincipal}{RETORNO AO VOLUME 2 - PARTE 2}}
\end{center}
\vspace{1em}

% ----------------------------------------------------
% BLOCO DO CAPÍTULO 3 (VOL 2)
% ----------------------------------------------------
\begin{tcolorbox}[
    colback=corclara!30,
    colframe=corprincipal,
    title=\textbf{Capítulo 3: Derivadas Direcionais e Gradiente (Vol. 2)},
    arc=2mm,
    boxrule=0.5mm,
    fontupper=\small
]
\begin{itemize}[label=$\square$, leftmargin=*, itemsep=0.8em]
    \item \textbf{Seção 3.1} (Regra da Cadeia): 1d, 1g, 1i, 1m, 1p, 2a, 2f, 2h, 5, 7, 8a, 8f
    \item \textbf{Seção 3.2} (Vetor Gradiente): 1, 4, 8, 9
    \item \textbf{Seção 3.3} (Derivada Direcional): 1a, 1c, 3a, 3c, 5
    \item \textbf{Seção 3.4} (Plano Tangente): 1b, 1c, 1d, 1h, 1j, 11, 2, 3b, 3d
\end{itemize}
\end{tcolorbox}

\vspace{2em}
\begin{center}
    {\Large\bfseries\textcolor{corprincipal}{RETORNO AO VOLUME 1 - PARTE 3}}
\end{center}
\vspace{1em}

% ----------------------------------------------------
% BLOCO DO CAPÍTULO 13
% ----------------------------------------------------
\begin{tcolorbox}[
    colback=corclara!30,
    colframe=corprincipal,
    title=\textbf{Capítulo 13: Aplicações da Integral},
    arc=2mm,
    boxrule=0.5mm,
    fontupper=\small
]
\begin{itemize}[label=$\square$, leftmargin=*, itemsep=0.8em]
    \item \textbf{Seção 13.1} (Cálculo de Áreas): 1a, 1d, 1g, 1j, 1m, 2
    \item \textbf{Seção 13.2} (Volumes de Sólidos de Revolução - Discos): 1a, 1d, 1g, 2c, 2d
    \item \textbf{Seção 13.3} (Volumes por Cascas Cilíndricas): 2, 3
    \item \textbf{Seção 13.4} (Comprimento de Arco): 1a, 1d
    \item \textbf{Seção 13.5} (Área de Superfície de Revolução): 1b, 1c, 1e
    \item \textbf{Seção 13.6} (Centro de Massa e Trabalho): 1b, 1c, 1e, 2
\end{itemize}
\end{tcolorbox}
